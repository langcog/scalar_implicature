% Template for Cogsci submission with R Markdown

% Stuff changed from original Markdown PLOS Template
\documentclass[10pt, letterpaper]{article}

\usepackage{cogsci}
\usepackage{pslatex}
\usepackage{float}

% amsmath package, useful for mathematical formulas
\usepackage{amsmath}

% amssymb package, useful for mathematical symbols
\usepackage{amssymb}

% hyperref package, useful for hyperlinks
\usepackage{hyperref}

% graphicx package, useful for including eps and pdf graphics
% include graphics with the command \includegraphics
\usepackage{graphicx}

% Sweave(-like)
\usepackage{fancyvrb}
\DefineVerbatimEnvironment{Sinput}{Verbatim}{fontshape=sl}
\DefineVerbatimEnvironment{Soutput}{Verbatim}{}
\DefineVerbatimEnvironment{Scode}{Verbatim}{fontshape=sl}
\newenvironment{Schunk}{}{}
\DefineVerbatimEnvironment{Code}{Verbatim}{}
\DefineVerbatimEnvironment{CodeInput}{Verbatim}{fontshape=sl}
\DefineVerbatimEnvironment{CodeOutput}{Verbatim}{}
\newenvironment{CodeChunk}{}{}

% cite package, to clean up citations in the main text. Do not remove.
\usepackage{cite}

\usepackage{color}

% Use doublespacing - comment out for single spacing
%\usepackage{setspace}
%\doublespacing


% % Text layout
% \topmargin 0.0cm
% \oddsidemargin 0.5cm
% \evensidemargin 0.5cm
% \textwidth 16cm
% \textheight 21cm

\title{The importance of negative alternatives in pragmatic implicature}


\author{{\large \bf Benjamin Peloquin} \\ \texttt{bpeloquin@stanford.edu} \\ Department of Psychology \\ Stanford University \And {\large \bf Michael C. Frank} \\ \texttt{mcfrank@stanford.edu} \\ Department of Psychology \\ Stanford University}

\begin{document}

\maketitle

\begin{abstract}
Frameworks of pragmatic enrichment, starting with Grice 1975, implicitly
assume the presence of salient alternatives. For example, the standard
quantity implicature arises the presence of a competiting and more
informative alternatives. Using a Bayesian formulzation of this theory
we assess the degree to which alternatives influence pragmatic
enrichment of scalar items. Our findings suggest that the presence of a
complete of set of possible alternatives.

\textbf{Keywords:}
pragmatics; scalar implicature; bayesian modeling
\end{abstract}

\section{Introduction}\label{introduction}

Implicature as a case study for pragmatics

RSA as a way of studying implicature ({\textbf{???}})

Scalar implicature in particular, what are the alternatives?
({\textbf{???}}) ({\textbf{???}})

Using the scalar diversity approach to get beyond the some/all
implicature ({\textbf{???}})

Outline of the rest of the paper

\section{Experiment 1a,b: Entailment
scales}\label{experiment-1ab-entailment-scales}

\subsection{Participants}\label{participants}

\subsection{Procedure}\label{procedure}

\subsection{Results and Discussion}\label{results-and-discussion}

\section{Experiment 2: What are the
alternatives?}\label{experiment-2-what-are-the-alternatives}

\subsection{Participants}\label{participants-1}

\subsection{Procedure}\label{procedure-1}

\subsection{Results and Discussion}\label{results-and-discussion-1}

\section{Experiment 3a,b: Adding Negative
alternatives}\label{experiment-3ab-adding-negative-alternatives}

\subsection{Participants}\label{participants-2}

\subsection{Procedure}\label{procedure-2}

\subsection{Results and Discussion}\label{results-and-discussion-2}

\section{Model}\label{model}

\subsection{Details}\label{details}

\subsection{Model fitting and
comparison}\label{model-fitting-and-comparison}

\section{General Discussion}\label{general-discussion}

\section{Acknowledgements}\label{acknowledgements}

Thanks to NSF BCS \#XYZ. Thanks to Michael Franke, Judith Degen, and
Noah Goodman.

\section{References}\label{references}

\setlength{\parindent}{-0.1in} \setlength{\leftskip}{0.125in} \noindent

\end{document}
