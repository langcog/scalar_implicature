% Template for Cogsci submission with R Markdown

% Stuff changed from original Markdown PLOS Template
\documentclass[10pt, letterpaper]{article}

\usepackage{cogsci}
\usepackage{pslatex}
\usepackage{float}

% amsmath package, useful for mathematical formulas
\usepackage{amsmath}

% amssymb package, useful for mathematical symbols
\usepackage{amssymb}

% hyperref package, useful for hyperlinks
\usepackage{hyperref}

% graphicx package, useful for including eps and pdf graphics
% include graphics with the command \includegraphics
\usepackage{graphicx}

% Sweave(-like)
\usepackage{fancyvrb}
\DefineVerbatimEnvironment{Sinput}{Verbatim}{fontshape=sl}
\DefineVerbatimEnvironment{Soutput}{Verbatim}{}
\DefineVerbatimEnvironment{Scode}{Verbatim}{fontshape=sl}
\newenvironment{Schunk}{}{}
\DefineVerbatimEnvironment{Code}{Verbatim}{}
\DefineVerbatimEnvironment{CodeInput}{Verbatim}{fontshape=sl}
\DefineVerbatimEnvironment{CodeOutput}{Verbatim}{}
\newenvironment{CodeChunk}{}{}

% cite package, to clean up citations in the main text. Do not remove.
\usepackage{cite}

\usepackage{color}

% Use doublespacing - comment out for single spacing
%\usepackage{setspace}
%\doublespacing


% % Text layout
% \topmargin 0.0cm
% \oddsidemargin 0.5cm
% \evensidemargin 0.5cm
% \textwidth 16cm
% \textheight 21cm

\title{Determining the alternatives for scalar implicature}


\author{{\large \bf Benjamin Peloquin} \\ \texttt{bpeloqui@stanford.edu} \\ Department of Psychology \\ Stanford University \And {\large \bf Michael C. Frank} \\ \texttt{mcfrank@stanford.edu} \\ Department of Psychology \\ Stanford University}

\begin{document}

\maketitle

\begin{abstract}
Successful communication regularly requires listeners to make pragmatic
inferences - enrichments beyond the literal meaning of a speaker's
utterance. For example, when interpreting a sentence such as ``Alice ate
some of the cookies,'' listeners routinely infer that Alice did not eat
all of them. A Gricean account of this phenomena assumes the presence of
alternatives (like ``all of the cookies'') with varying degrees of
informativity, but it remains an open question precisely what these
alternatives are. We collect empirical measurements of speaker and
listener judgments and use these as inputs to a computational model of
pragmatic inference. This approach allows us to test hypotheses about
how well different sets of alternatives predict implicature performance
across a range of different scales. Our findings suggest that
comprehenders likely consider a much broader set of alternatives beyond
those entailed by the initial description.

\textbf{Keywords:}
pragmatics; scalar implicature; bayesian modeling
\end{abstract}

\section{Introduction}\label{introduction}

How much of what we mean comes from the words that go unsaid? As
listeners, our ability to make precise inferences about a speaker's
intended meaning in context is indispensable to successful
communication. For example, listeners commonly enrich the meaning of the
scalar item \emph{some} to \emph{some but not all} in sentences like
``Alice ate some of the cookies'' (Grice, 1975; Horn, 1984; Levinson,
2000). These inferences, called \emph{scalar implicatures}, have been an
important test case for understanding pragmatic inferences more
generally. A Gricean account of this phenomenon assumes listeners reason
about the meaning the speaker intended by incorporating knowledge about
a) alternative scalar items a speaker could have used (such as
\emph{all}) and b) the relative informativity of using such alternatives
(Grice, 1975). According to this account, a listener will infer that the
speaker must have intended that Alice did not eat all the cookies
because otherwise it would have been underinformative to use the
descriptor \emph{some} when the alternative \emph{all} could have been
used just as easily.

But what are the alternatives that should be considered in the
implicature computation more generally? Under classic accounts,
listeners consider only those words whose meanings entail the actual
message (Horn, 1972), and these alternatives enter into conventionalized
or semi-conventionalized scales (Levinson, 2000). For example, because
\emph{all} entails \emph{some}, and hence is a ``stronger'' meaning,
\emph{all} should be considered as an alternative to \emph{some} in
implicatures. Similar scales exist for non-quantifier scales, e.g.
\emph{loved} entails \emph{liked} (and hence ``I liked the movie''
implicates that I didn't love it).

Recent empirical evidence has called into question whether entailment
scales are all that is necessary for understanding scalar implicature.
For example, Degen \& Tanenhaus (2015) demonstrated that the scalar item
\emph{some} was judged less appropriate when exact numbers were seen as
viable alternatives. And in a different paradigm, Tiel (2014) found
converging evidence that \emph{some} was judged to be atypical for small
quantities. These data provide indirect evidence about a broader set of
alternatives: since \emph{some} is logically true of sets with one or
two members, these authors argued that the presence of salient
alternatives (the words \emph{one} and \emph{two}, for example) reduced
the felicity of \emph{some} via a pragmatic inference.

By formalizing pragmatic reasoning, computational models can help
provide more direct evidence about the role that alternatives play. The
``rational speech act'' model (RSA) is one recent framework for
understanding inferences about meaning in context (Frank \& Goodman,
2012; N. D. Goodman \& Stuhlm{ü}ller, 2013). RSA models frame language
understanding as a special case of social cognition, in which listeners
and speakers reason recursively about one another's goals. In the case
of scalar implicature, a listener makes a probabilistic inference about
what the speaker's most likely communicative goal was, given that she
picked the quantifier \emph{some} rather than the stronger quantifier
\emph{all}. In turn, the speaker reasons about what message would best
convey her intended meaning to the listener, given that he is reasoning
in this way. This recursion is grounded in a ``literal'' listener who
reasons only according to the basic truth-functional semantics of the
language.

Franke (2014) used an RSA-style model to assess what alternatives a
speaker would need to consider in order to produce the
typicality/felicity ratings reported by Degen \& Tanenhaus (2015) and
Tiel (2014). In order to do this, Franke (2014)'s model assigned weights
to a set of alternative numerical expressions. Surprisingly, along with
weighting \emph{one} highly (a conclusion that was supported by the
empirical work), the best-fitting model assigned substantial weight to
\emph{none} as an alternative. This finding was especially surprising
considering the emphasis of standard theories on scalar items that stand
in entailment relationships with one another (e.g. \emph{one} entails
\emph{some} even if it is not classically considered to be part of the
scale).

In our current work, we pick up where these previous studies left off,
considering the set of alternatives for implicature using the RSA model.
To gain empirical traction on this issue, however, we broaden the set of
scales we consider. Our inspiration for this move comes from work by Van
Tiel, Van Miltenburg, Zevakhina, \& Geurts (2014), who examined a
phenomenon that they dubbed ``scalar diversity,'' namely the substantial
difference in the strength of scalar implicature across a variety of
scalar pairs (e.g. \emph{liked/loved,} or \emph{palatable/delicious.}).
Making use of some of this diversity allows us to investigate the ways
that different alternative sets give rise to implicatures of different
strengths across scales.

We begin by presenting the computational framework we use throughout the
paper. We next describe a series of experiments designed to measure both
the literal semantics of a set of scalar items and comprehenders'
pragmatic judgments for these same items. These experiments allow us to
compare the effects of different alternative sets on our ability to
model listeners' pragmatic judgments. To preview our results: we find
that standard entailment alternatives do not allow us to fit
participants' judgments, but that expanding the range of alternatives
empirically (by asking participants to generate alternative messages)
allows us to model listener judgments with high accuracy.

\section{Modeling Implicature Using
RSA}\label{modeling-implicature-using-rsa}

We begin by giving a brief presentation of the basic RSA model. This
model simulates the judgments of a pragmatic listener who wants to infer
a speaker's intended meaning \(m\) from her utterance \(u\). For
simplicity, we present a version of this model in which there is only
full recursion: that is, the pragmatic listener reasons about a
pragmatic speaker, who in turn reasons about a ``literal listener.'' We
assume throughout that this computation takes place in a signaling game
(Lewis, 1969) with a fixed set of possible meanings \(m \in M\) and a
fixed possible set of utterances \(u \in U\), with both known to both
participants. Our goal in this study is to determine what utterances
fall in \(U\).

In the standard RSA model, the pragmatic listener (denoted \(L_1\)),
makes a Bayesian inference:

\[p_{L_1}(m \mid u) \propto p_{S_1} (u \mid m) p(m) \tag{1}\]

\noindent In other words, the probability of a particular meaning given
an utterance is proportional to the speaker's probability of using that
particular utterance to express that meaning, weighted by a prior over
meanings. This prior represents the listener's \emph{a priori}
expectations about plausible meanings, independent of the utterance.
Because our experiments take place in a context in which listeners
should have very little expectation about which meanings speakers want
to convey, for simplicity we assume a uniform prior \(p(m) \propto 1\).

The pragmatic speaker in turn considers the probability that a literal
listener would interpret her utterance correctly:

\[p_{S_1}(u \mid m) \propto p_{L_0} (m \mid u)\]

\noindent where \(L_0\) refers to a listener who only considers the
truth-functional semantics of the utterance (that is, which meanings the
utterance can refer to).

This model of the pragmatic speaker (denoted \(S_1\)) is consistent with
a speaker who chooses words to maximize the utility of an utterance in
context (Frank \& Goodman, 2012), where utility is operationalized as
the informativity of a particular utterance (surprisal) minus a cost:

\[p_{S_1}(u \mid m) \propto e^{-\alpha[-log(p_{L_0}(m \mid u)) - C(u)]}\]

\noindent where \(C(u)\) is the cost of a particular utterance,
\(-log(p_{L_0})\) represents the \emph{surprisal} of the message for the
literal listener (the information content of the utterance), and
\(\alpha\) is a parameter in a standard choice rule. If \(\alpha=0\),
speakers choose randomly and as \(\alpha \rightarrow \infty\), they
greedily choose the highest probability alternative. In our simulations
below, we treat \(\alpha\) as a free parameter and fit it to the data.

To instantiate our signaling game with a tractable message set \(M\), in
our studies we adopt the world of restaurant reviews as our
communication game. We assume that speakers and listeners are trying to
communicate the number of stars in an online restaurant review (where
\(m \in \{1, 2, 3, 4, 5\}\)). We then use experiments to measure three
components of the model. First, to measure literal semantics
\({p_{L_0} (m \mid u)}\) (we ask experiment participants to judge
whether a message is compatible with a particular meaning (Experiment
1). Second, to generate a set of plausible alternative messages in
\(U\), we elicit alternatives directly (Experiment 2). Lastly, to obtain
human \(L_1\) pragmatic judgments, we ask participants to interpret a
speaker's utterances (Experiment 3).

\section{Experiment 1: Literal listener
task}\label{experiment-1-literal-listener-task}

\begin{CodeChunk}
\begin{figure}[t]
\includegraphics{figs/allScalesTable-1} \caption[Stimuli for Experiments 1, 2, and 3]{Stimuli for Experiments 1, 2, and 3}\label{fig:allScalesTable}
\end{figure}
\end{CodeChunk}

\begin{CodeChunk}
\begin{figure*}[t]

{\centering \includegraphics{figs/stimuli_exp1-1} 

}

\caption[(Left) A trial from Experiment 1 (literal listener) with the target scalar `liked]{(Left) A trial from Experiment 1 (literal listener) with the target scalar `liked.' (Right) A trial from Experiment 3 (pragmatic listener) with the target scalar `liked.'}\label{fig:stimuli_exp1}
\end{figure*}
\end{CodeChunk}

Experiment 1 was conducted to approximate literal listener semantic
distributions \(p_{L_0}(m \mid u)\) for five pairs of scalar items taken
from Tiel (2014). We include three conditions in Experiment 1,
corresponding to the sets of alternatives within a scale that
participants were presented with: two alternatives (``entailment''),
four alternatives, and five alternatives. The two alternatives condition
makes a test of the hypothesis that the two members of the classic Horn
(entailment) scale (Horn, 1972) are the only alternatives necessary to
predict the strength of listeners' pragmatic inference. The four and
five alternatives conditions then add successively more alternatives to
test whether including a larger number of alternatives will increase
model
fit.\footnote{Note that alternatives in the four and five alternatives conditions were chosen on the basis of Experiment 2, which was run chronologically after the two-alternative condition; all literal listener experiments are grouped together for simplicity in reporting.}
A secondary goal of Experiment 1 is to test whether the set of
alternatives queried during literal semantic elicitation impacts
compatibility judgments. (If it does we should see differences in
compatibility judgments for shared items between experiments.)

\subsection{Methods}\label{methods}

\subsubsection{Participants}\label{participants}

Conditions were run sequentially. In each condition we recruited 30
participants from Amazon Mechanical Turk (AMT). In the two alternative
condition, 16 participants were excluded for either failing to pass two
training trails or were not native English speakers, leaving a total
sample of 14
participants.\footnote{The majority of respondent data excluded from the two alternative condition was caused by failure to pass training trials. We believe the task may have been too difficult for most respondents and made adjustments to the training trials in later conditions.}
In the four alternative condition, 7 participants were excluded for
either failing to pass two training trials or were not native English
speakers, leaving a total sample of 23 participants. In the five
alternative condition, 3 participants were excluded for either failing
to pass two training trials or were not native English speakers, leaving
a total sample of 27.

\subsubsection{Design and procedure}\label{design-and-procedure}

Figure \ref{fig:stimuli_exp1}, left, shows the experimental setup.
Participants were presented with a target scalar item and a star rating
(1--5 stars) and asked to judge the compatibility of the scalar item and
star rating. Compatibility was assessed through a binary ``yes/no''
response to a question of the form, ``Do you think that the person
thought the food was \_\_\_\_?" where a target scalar was presented in
the blank. Each participant saw all scalar item and star rating
combinations for their particular condition, in a random order.

The two alternatives condition included only the scalar pairs from Tiel
(2014). The four alternatives condition included the two scalar items
plus the top two alternatives generated for each scalar family by
participants in Experiment 2. The five alternatives condition included
the four previous items plus one more neutral item chosen from those
alternatives generated in Experiment 2. See Figure
\ref{fig:allScalesTable} for the complete list of alternatives used in
each condition.

\subsection{Results and Discussion}\label{results-and-discussion}

\begin{CodeChunk}
\begin{figure*}[t]

{\centering \includegraphics{figs/exp1Plots-1} 

}

\caption[Literal listener judgments from Experiments 1a,b,c]{Literal listener judgments from Experiments 1a,b,c. Proportion of participants indicating compatibility (answering 'yes') is shown on the vertical axis, with the horizontal axis showing number of stars on which the utterance was judged. Rows are grouped by scale and items within rows are ordered by valence. Colors indicate the specific experiment (1a,b,c) with experiments including different numbers of items.}\label{fig:exp1Plots}
\end{figure*}
\end{CodeChunk}

Figure \ref{fig:exp2Plots} plots literal listener \(p_{L0}(m|u)\) scalar
item distributions for the three conditions. Each row shows a unique
scalar family with items ordered horizontally by valence. Several trends
are visible. First, in each scale, the alternatives spanned the star
scale, such that there were alternatives that were highly compatible
with both the lowest and highest numbers of stars. Second, there was
clear variability between scalar families. For example, compatibility
judgments for the top items in the \emph{memorable / unforgettable}
scale were more similar than those for \emph{good / excellent} or
\emph{liked / loved}. Finally, there was substantial consistency in
ratings for items that were repeated across experiments, suggesting that
this paradigm elicited stable judgments from participants.

To test our secondary hypothesis, that the different sets of scalar
items might result in differences in compatibility judgments between
experiments, we ran a mixed effects model. We regressed compatibility
judgments on scale, number of stars and condition, with subject and word
level random effects, which was the maximal structure that converged.
Results indicate no significant differences between Experiment 1a and
1c, \(b =\)-0.05, \(Z =\) -0.53, \(p =\) 0.59 or between 1b and 1c,
\(b =\) -0.04, \(Z =\) -0.52 \(p =\) 0.6 and the addition of condition
as a predictor did not significantly improve model fit when compared to
a model without the experiment variable using ANOVA \(\chi^2(\) 2
\() =\) 0.43, \(p =\) 0.81.

\section{Experiment 2: Alternative
Elicitation}\label{experiment-2-alternative-elicitation}

To elicit empirical alternatives for the scales we used in Experiment 1,
we adopted a modified cloze task inspired by Experiment 3 of Tiel
(2014).

\subsection{Methods}\label{methods-1}

\subsubsection{Participants}\label{participants-1}

We recruited 30 workers on AMT. All participants were native English
speakers and naive to the purpose of the experiment.

\subsubsection{Design and procedure}\label{design-and-procedure-1}

Participants were presented a target scalar item from our original
entailment set (see Figure \ref{fig:allScalesTable}) embedded in a
sentence such as, ``In a recent restaurant review someone said they
thought they the food was \_\_\_\_," with a target scalar presented in
the blank. Participants were then asked to generate plausible
alternatives by responding to the question, ``If they'd felt differently
about the food, what other words could they have used instead of
\_\_\_\_?" They were prompted to generate three unique alternatives.

\subsection{Results and Discussion}\label{results-and-discussion-1}

\begin{CodeChunk}
\begin{figure}[t]

{\centering \includegraphics{figs/exp2_altsPlot_likedLoved-1} 

}

\caption[Combined counts for participant-generated alternatives for the 'liked  loved' scale in Experiment 2]{Combined counts for participant-generated alternatives for the 'liked  loved' scale in Experiment 2.}\label{fig:exp2_altsPlot_likedLoved}
\end{figure}
\end{CodeChunk}

\begin{CodeChunk}
\begin{figure*}[t]

{\centering \includegraphics{figs/exp2Plots-1} 

}

\caption[Pragmatic listener judgements for scalar items]{Pragmatic listener judgements for scalar items. Proportion of participants generating a star rating is shown on the vertical axis, with the horizontal axis showing number of stars on which the utterance was judged. Line type denotes condition, and colors indicate the particular scalar items. Each panel shows one scalar pair, with only entailment items (two alternatives condition) shown here for simplicity.}\label{fig:exp2Plots}
\end{figure*}
\end{CodeChunk}

Figure \ref{fig:exp2_altsPlot_likedLoved} shows an example alternative
set for the scalar items \emph{liked} and \emph{loved} (combined).
Alternative distributions for the other scalar pairs (e.g..
\emph{good/excellent}, \emph{memorable/unforgettable}) were similarly
long-tailed.

\section{Experiment 3: Pragmatic
Listener}\label{experiment-3-pragmatic-listener}

Experiment 3 was conducted to measure pragmatic
judgments---\(p_{L_1}(m \mid u)\) in Equation \((1)\). As in Experiment
1, we include several conditions to test inferences in the presence of
different alternative sets. In the two alternatives condition,
participants made judgments for items included in the entailment scales.
In the four alternatives condition, participants made judgments for the
entailment items and also the top two alternatives elicited for each
scale in Experiment 2. Including two conditions with differing
alternatives allowed us to rule out the potential effects of having a
larger set of alternatives during the pragmatic judgment elicitation and
also provided two sets of human judgments to compare with model
predictions.\footnote{Note that alternatives in the four alternatives condition were chosen on the basis of Experiment 2, which was run chronologically after the two alternatives condition; both pragmatic listener experiments are grouped together for simplicity in reporting.}

\subsection{Participants}\label{participants-2}

We recruited 100 participants from AMT, 50 for each condition. Data for
9 participants was excluded from the two alternatives condition after
participants either failed to pass two training trials or were
non-native English speakers, leaving a total sample of 41 participants.
In the four alternative condition, data from 7 participants was excluded
after participants either failed to pass two training trials or were not
native English speakers, leaving 43 participants.

\subsection{Procedure}\label{procedure}

Participants were presented with a one-sentence prompt containing a
target scalar item such as ``Someone said they thought the food was
\_\_\_\_\_.'' Participants were then asked to generate a star rating
representing the rating they thought the reviewer likely gave. Each
participant was presented with all scalar items in a random order. The
experimental setup is shown in Figure \ref{fig:stimuli_exp1}, right.

\subsection{Results and Discussion}\label{results-and-discussion-2}

Figure \ref{fig:exp2Plots} plots pragmatic listener judgments
distributions for ``weak'' / ``strong'' scalar pairs (e.g.
\emph{good}/\emph{excellent}). We include only the original entailment
pairs in this figure for simplicity. Several trends are visible. First,
in each scale participants generated implicatures. They were
substantially less likely to assign high star ratings to weaker scalar
terms, despite the literal semantic compatibility of those terms with
those states shown in Experiment 1. Second, the size of the difference
between strong and weak scalar items varied considerably across scales,
consistent with previous work (Tiel, 2014).

To test our secondary hypothesis, that the different sets of scalar
items in the two conditions might result in differences in pragmatic
judgments for the same items, we fit a mixed effects model. We regressed
pragmatic judgments on scale and condition with subject- and word-level
random effects, which was the maximal structure that converged. There
were no significant differences between conditions, \(b =\) 0.05, \(t(\)
150 \() =\) 1.04, \(p =\) 0.3 and the addition of the condition
predictor did not significantly improve model fit when compared to a
model without that variable \(\chi^2(\) 1 \() =\) 1.13, \(p =\) 0.29.

\section{Model}\label{model}

Using literal listener data from Experiment 1, we conducted a set of
simulations with the RSA model. Each simulation kept the model constant,
fitting the choice parameter \(\alpha\) as a free parameter, but used a
set of alternatives to specify the scale over which predictions were
computed. We considered four different alternative sets, with empirical
measurements corresponding to those shown in Table
\textbackslash{}ref\{fig:allScalesTable): 1) the two alternatives in the
classic entailment scales, 2) those two alternatives with the addition
of a generic negative alternative, 3) the expanded set of four
alternatives tested in Experiment 1, and 4) the expanded set of five
alternatives tested in Experiment 1.

Model fit with human judgments was significantly improved by the
inclusion of alternatives beyond the entailment items (Table
\ref{fig:xtable}). The two alternatives model contained only entailment
items, which, under classic accounts, should be sufficient to generate
implicature, but fit to data was quite poor with these items. The
addition of a generic negative element produced some gains in
performance, but much higher performance was found when we included four
and five alternatives, with the alternatives derived empirically for the
specific scale we used. An example fit for the five-alternative model is
shown in Figure \ref{fig:fiveAltsScatter}.

\begin{table}[ht]
\centering
\begin{tabular}{lrrr}
  \hline
Model & $\alpha$ & Exp. 3a & Exp. 3b \\ 
  \hline
Two alts &   9 & 0.54 & 0.57 \\ 
  Two alts + generic negative  &   6 & 0.62 & 0.66 \\ 
  Four alts &   4 & 0.84 & 0.90 \\ 
  Five alts &   4 & 0.86 & 0.90 \\ 
   \hline
\end{tabular}
\caption{Model performance with fitted alpha levels. Model fit assessed through correlation with human judgments in our two Pragmatic listener experiments (3a,b)} 
\end{table}

\begin{CodeChunk}
\begin{figure*}[t]

{\centering \includegraphics{figs/fiveAltsScatter-1} 

}

\caption[Judgments from Experiment 3, four alternatives condition, are plotted against model predictions]{Judgments from Experiment 3, four alternatives condition, are plotted against model predictions. Colors show the star rating for individual judgments.}\label{fig:fiveAltsScatter}
\end{figure*}
\end{CodeChunk}

\section{General Discussion}\label{general-discussion}

Pragmatic inference requires reasoning about alternatives. The
fundamental pragmatic computation is counterfactual: ``if she had meant
X, she would have said Y, but she didn't.'' Yet the nature of these
alternatives has been controversial. For a few well-studied scales, a
small set of logically-determined alternatives has been claimed to be
all that is necessary (Horn, 1972). For other, contextually-determined
inferences, the issue of alternatives has been considered relatively
intractable in terms of formal inquiry (Sperber \& Wilson, 1995).

In our current work, we used the rational speech act framework to
investigate the set of alternatives that best allowed the model to
predict pragmatic judgments across a series of different scales. We
found that the best predictions came when a range of scale-dependent
negative and neutral alternatives were added to the implicature
computation, suggesting the importance of considering non-entailment
alternatives. This work builds on previous investigations, reinforcing
the claim that negative alternatives are critical for understanding
implicature (Franke, 2014), and replicates and extends findings that
different lexical scales produce strikingly different patterns of
inference ({\textbf{???}}).

While improvements in model fit were substantial as we moved from two to
four alternatives, we saw only a minor increase in fit from the move to
five alternatives. One possible explanation is that alternatives are
differentially salient in context, and moving to larger sets would
consider weighting the alternatives differentially (as Franke, 2014
did). Preliminary simulations using weightings derived from Experiment 2
provide some support for this idea but would require further empirical
work for confirmation.

The precise set of alternatives present during implicature is likely to
be domain dependent. Our current empirical paradigm elicited literal
semantics, pragmatic judgments, and plausible alternatives all within
the restricted domain of restaurant reviews. Our measurements might have
differed substantially if we had instead grounded our ratings in a
different context. Future investigations should probe the
context-specificity of the weight and availability of particular
alternative sets.

More broadly, considering the context- and domain-specificity of
alternative sets may provide a way to unite what Grice (1975) called
``generalized'' (cross-context) and ``particularized''
(context-dependent) implicatures. Rather than being grounded in a firm
distinction, we may find that these categories are simply a reflection
of the effects of context on a constantly-shifting set of pragmatic
alternatives.

\section{Acknowledgements}\label{acknowledgements}

Thanks to NSF BCS \#1456077 for support, and thanks to Michael Franke,
Judith Degen, and Noah Goodman for valuable discussion.

\section{References}\label{references}

\setlength{\parindent}{-0.1in} \setlength{\leftskip}{0.125in} \noindent

Degen, J., \& Tanenhaus, M. K. (2015). Processing scalar implicature: A
constraint-based approach. \emph{Cognitive Science}, \emph{39}(4),
667--710.

Frank, M., \& Goodman, N. (2012). Predicting pragmatic reasoning in
language games. \emph{Science}, \emph{336}(6084), 998.

Franke, M. (2014). Typical use of quantifiers: A probabilistic speaker
model. In \emph{Proceedings of the 36th annual conference of the
cognitive science society} (pp. 487--492).

Goodman, N. D., \& Stuhlm{ü}ller, A. (2013). Knowledge and implicature:
Modeling language understanding as social cognition. \emph{Topics in
Cognitive Science}, \emph{5}(1), 173--184.

Grice, H. P. (1975). Logic and conversation. In P. Cole \& J. Morgan
(Eds.), \emph{Syntax and semantics} (Vol. 3). New York: Academic Press.

Horn, L. R. (1972). \emph{On the semantic properties of logical
operators.} (PhD thesis). University of California, Los Angeles.

Horn, L. R. (1984). Toward a new taxonomy for pragmatic inference:
Q-based and R-based implicature. \emph{Meaning, Form, and Use in
Context}, \emph{42}.

Levinson, S. C. (2000). \emph{Presumptive meanings: The theory of
generalized conversational implicature}. MIT Press.

Lewis, D. (1969). \emph{Convention: A philosophical study}. John Wiley
\& Sons.

Sperber, D., \& Wilson, D. (1995). \emph{Relevance: Communication and
cognition} (2nd ed.). Oxford, UK: Blackwell.

Tiel, B. van. (2014). Quantity matters: Implicatures, typicality, and
truth.

Van Tiel, B., Van Miltenburg, E., Zevakhina, N., \& Geurts, B. (2014).
Scalar diversity. \emph{Journal of Semantics}, ffu017.

\end{document}
